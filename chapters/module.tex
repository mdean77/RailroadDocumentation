% This document contains site monitoring information for the relevant network
%
%  				J. Michael Dean, M.D.
%  				University of Utah School of Medicine

% The following line makes this document point back so that my software will synchronize
% between the preview and source windows.

%!TEX root = ../Railroad.tex

\chapter{Wasatch N Scale Module}
\minitoc
%\index{test}
\section{Wasatch N Scale Club}
I joined the Salt Lake City based railroad club that later became the NRMA Northern Division, and became aware that there was
a club devoted entirely to N scale modeling.  In the spring of 2013 I contacted their president, Bob Gerald, and arranged to come to the Evanston train show that was being held in July.  I was invited to bring some trains, as the club would be set up and they needed people to run train.  I didn't really understand that, but my wife and I drove up for an afternoon and ran trains around the club layout.\\

What I really did not understand was that they really \emph{needed} interested people to run trains.  Having subsequently attended numerous train shows, including several in Evanston, it becomes important to make sure that trains are running throughout the show, because the public (especially children) simply love to watch trains go.\\

At my first club meeting, I heard about the club standards, and learned a fair amount about N-trak (upon which our clubs standards are based).  It was pointed out that full membership required building a module, since full members could vote on changing module standards and it was considered unreasonable to vote on such matters unless you built a module yourself.  Modules range in size, but I was informed that the club really needed some more basic two by four foot modules, rather than additional large or corner modules.  Thus, I embarked on the adventure of building my first module.

\section{Initial Module Planning}


\section{Building a Mountain in the Kitchen}

\section{Wiring and Tortoises}

\section{Adding Signaling to the Module}

\section{Fixing a Disaster}

\section{Status 2019 - 2020}

\section{Conclusions about Modules and Clubs}
Modules are an excellent way for a beginning or advanced modeler to get something meaningful off the ground.  When I started my module, I had only a rudimentary oval layout in my basement, and it was simply not tenable to learn about all the various technologies quickly in the context of a basement layout.  The club activities at train shows provided me with a timeline, including deadlines, and this led to sporadic but regular progress on the module.  I have also used the module as an experimental platform for different scenery techniques, such as static grass.\\

I cannot say enough how important a club can be to a beginner model railroader.  The direct hands on assistance I received by Robert Arneson and Bob Gerald was critical - this enabled me to have my module ready for the first Hostler show, complete with Tortoise controlled turnouts and rudimentary scenery.  In one afternoon, they propelled me through what would have taken months without their assistance.